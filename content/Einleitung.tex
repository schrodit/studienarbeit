\chapter{Einleitung}

    Was bis vor kurzer Zeit nur ein wissenschaftlicher Teil der Informatik war, erhält nun immer größere Bedeutung und Einfluss in vielen weiteren wirtschaftlichen und wissenschaftlichen Themen. 
    Seit große IT Firmen wie Google oder Facebook große Fortschritte mit maschinellem Lernen und künstlicher Intelligenz erzielen, wird maschinelles Lernen auch produktiv eingesetzt und immer mehr Nutzer kommen damit im täglichen Leben in Berührung.
    So liegt es Nahe, dies auch auf bisher unberührten Branchen wie die Automobil- oder Elektronikbranche auszuweiten.\\
    \newline
    Dazu gehört auch die elektrische Energiewirtschaft, welche unter anderem die elektronische Infrastruktur und somit auch die Grundversorgung an elektrischer Energie bereitstellt.
    Die Versorgung von privaten Haushalten sowie Firmen mit elektrischer Energie ist eine nicht mehr wegzudenkende Kernindustrie.
    Durch neue Datencenter und den Ausbau von digitalen Systemen wächst die Nachfrage nach diesem Rohstoff immer schneller wodurch für diese Branche eine enormer Wachstumchance besteht.
    \newline

    In dieser Arbeit wird Data-Mining auf elektrotechnische Größen angewendet um weiterführende semantische Aussagen über diese Werte zu erhalten. 
    Es werden verschiedene Netzdaten und dazugehörige Verläufe aufgezeichnet und mithilfe von verschiedenen Methoden von Maschinellem Lernen analysiert.
    Hauptbestandteil ist die Mustererkennung in den aufgenommenen Verläufen und deren Zuordnung zu verschiedenen Geräten in einem Haushalt. 
    Hierbei sollen verschiedene, normale Haushaltgeräte, wie Kaffeemaschinen oder Fernseher, welche innerhalb eines Stromnetzes eines privaten Haushalts betrieben werden, erkannt werden.
    Mit den damit erhobenen Daten können somit Aussagen über Laufleistungen getroffen werden.
    \newline
    
    Außerdem wird auf die Wirtschaftlichkeit des resultierenden Ergebnisses, sowie deren Produktivbetrieb eingegangen.\\
    \newline
    Für die maschinelle Analyse wird die Keras-API\footnote{https://keras.io/} mit Tensorflow im Backend verwendet.
    Die Datenverarbeitung, Visualisierung sowie die Trainings- und Testphasen werden mit Python\footnote{https://www.python.org/} umgesetzt.
