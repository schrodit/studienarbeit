\chapter{Einleitung}

    Was bis vor kurzer Zeit nur ein wissenschaftlicher Teil der Informatik war, erhält nun immer größere Bedeutung und Einfluss in vielen weiteren wirtschaftlichen und wissenschaftlichen Themen. 
    Seit Google, Facebook, etc. große Fortschritte in diesem Bereich erzielen und Maschinelles Lernen auch in produktivem Betrieb Einsatz findet, liegt es Nahe dies auch auf bisher unberührten Branchen auszuweiten.\\
    \newline
    In dieser Arbeit wird Data-Mining auf elektrotechnische Größen angewendet um weiterführende semantische Aussagen über diese Werte zu erhalten. 
    Es werden verschiedene Netzdaten und dazugehörige Verläufe aufgezeichnet und mithilfe von verschiedenen Methoden von Maschinellem Lernen analysiert.
    Hauptbestandteil ist die Mustererkennung in den aufgenommenen Verläufen und deren Zuordnung zu verschiedenen Geräten in einem Haushalt. 
    Hierbei sollen verschiedene normale Haushaltgeräte, wie Kaffeemaschinen oder Fernseher, welche in einem Stromnetzes eines privaten Haushalts erkannt werden um somit Aussagen über Laufleistungen zu treffen.
    Außerdem wird auf die wirtschaftliche Nutzung des resultierenden Modells, sowie deren Produktivbetrieb eingegangen.\\
    \newline
    Für die maschinelle Analyse wird die Keras-API\footnote{https://keras.io/} mit Tensorflow im Backend verwendet.
    Die Datenverarbeitung, Visualisierung sowie die Trainings- und Testphasen werden mit Python\footnote{https://www.python.org/} umgesetzt.
