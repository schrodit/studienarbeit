\chapter{Einleitung}

\section{Aufgabe}

\section{Architektur}

    Um Aussagen und KLassifikationen über ein Stromnetz bzw. die Geräte in einem Stromnetz machen zu können, werden zunächst viele Daten benötigt.
    Die Daten bestehen aus verschiedenen physikalische Größen, die zu einem gewissen Zeitpunkt in einem Stromnetz vorkommen.
    Zu diesen Größen gehört die allgemeine Netzspannung, die Netzfrequenz sowie sieben verschiedene Oberwellen (vgl. \ref{physikalischeGrundlagen}).
    Um einen allgemeinen Überblick über den Verlauf der Netzaktivität zu erhalten sowie verschiedene Zeiten und Geräte vergleichen zu können, müssen Daten über lange Zeiträume erhoben werden.\\
    \newline
    Zur Erhebung der Werte zur Netzaktivität wurde ein WeSense Messgerät\footnote{http://www.wesense-app.com/home-en/} verwendet.
    Dieses Gerät misst alle benötigten Werte und sendet diese über einen MQTT Broker an einen Service von WeSense, welcher dann die Daten aufwertet und in einer MSSQL Datenbank abspeichert.
    Die Werte werden sekündlich gemessen und in die Datenbank gespeichert, weshalb zunächst in eine row-based Datenbank gespeichert wird und später dann die Daten in eine column-based Datenbank zur schneller Analyse überführt werden.
    \newline

    Architektur Bild!!

    \subsection{Klassifikation der Messdaten}