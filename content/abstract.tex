\begin{abstract}

    Das Ziel dieser Studienarbeit war es mit Hilfe von verschiedenen Machine Learning-Methoden aus Stromdaten eines Haushalts verschiedene Geräte zu klassifizieren.
    Dazu wurden mit einem Messgeräte die allgemeine Spannung, Frequenz und verschiede Oberwellen eines üblichen Stromnetzwerks eines privat Haushalts über mehrere Monate erfasst.
    Hinzu wurden manuell verschiede Geräte wie eine Kaffemschine oder eine Mikrowelle klassifiziert.
    Anhand der Stromverläufe und den dazu klassifizierten Geräten wurden verschiedene neuronale Netzte trainiert und miteinander verglichen.
    Außerdem wurde versucht mit unüberwachten Lernen Gemeinsamkeiten innerhalb des Stromverlaufs ohne vorherige Klassifizierung zu erkennen.\\
    \newline
    Auch wurde die wirtschaftlichkeit sowie der produktive Einsatz der Ergebnisse beachtet.
    Die neuronalen Netze, welche die besten Ergebnisse erzielten werden außerdem im produktiv Betrieb getestet und eingesetzt.



\end{abstract}