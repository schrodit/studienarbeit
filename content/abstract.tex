\begin{abstract}

    Das Ziel dieser Studienarbeit ist es mithilfe verschiedener Machine Learning-Methoden aus Stromdaten eines Haushalts Geräte zu klassifizieren.
    Diese Klassifikation beinhaltet die Bestimmung der Laufzeit dieser Geräte innerhalb einer Zeitreihe. 
    Die Bestimmung der Laufzeit stellt dar, welche Geräte zu welchen Zeiten aktiv waren.
    Dazu wurden mit einem Messgerät die allgemeine Spannung, Frequenz und verschiedene Oberwellen eines üblichen Stromnetzwerks eines Privathaushalts über mehrere Monate erfasst.
    Hinzu wurden manuell verschiedene Geräte wie eine Kaffeemaschine oder eine Mikrowelle klassifiziert.
    Anhand der Stromverläufe und den dazu klassifizierten Geräten wurden verschiedene neuronale Netze trainiert und miteinander verglichen.\\
    \newline
    Auch wird die Wirtschaftlichkeit sowie der produktive Einsatz der Ergebnisse beachtet.
    Die neuronalen Netze, welche die besten Ergebnisse erzielten, werden außerdem im Produktivbetrieb getestet und eingesetzt.

\end{abstract}