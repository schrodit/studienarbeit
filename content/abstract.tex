\begin{abstract}

    Das Ziel dieser Studienarbeit ist es mithilfe von verschiedenen Machine Learning-Methoden aus Stromdaten eines Haushalts verschiedene Geräte zu klassifizieren.
    Diese Klassifikation beinhaltet die Bestimmung der Laufzeit dieser Geräte innerhalb einer Zeitreihe. 
    Die Bestimmung der Laufzeit bedeutet, dass bestimmt werden soll zu welchen Zeiten welche Geräte aktiv waren.
    Dazu wurden mit einem Messgeräte die allgemeine Spannung, Frequenz und verschiedene Oberwellen eines üblichen Stromnetzwerks eines privat Haushalts über mehrere Monate erfasst.
    Hinzu wurden manuell verschiedene Geräte wie eine Kaffeemaschine oder eine Mikrowelle klassifiziert.
    Anhand der Stromverläufe und den dazu klassifizierten Geräten wurden verschiedene neuronale Netze trainiert und miteinander verglichen.\\
    \newline
    Auch wird die wirtschaftlichkeit sowie der produktive Einsatz der Ergebnisse beachtet.
    Die neuronalen Netze, welche die besten Ergebnisse erzielten werden außerdem im produktiv Betrieb getestet und eingesetzt.

\end{abstract}