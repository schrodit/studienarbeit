\chapter{ProduktivBetrieb}
    Das in Kapitel \ref{Ergebnis} resultierende neuronale Netz sowie die Auswertung dessen Ergebnis kann bisher nur begrenzt genutzt werden.
    Um diese Funktionen produktiv einsetzten zu können wird eine Schnittstelle zur einfachen Verwendung benötigt.
    An diese Schnittstelle werden folgende weitere Vorraussetzungen gestellt.
    \begin{description}
        \item[Echzeitanalyse] Es soll möglich sein Daten beziehungsweise Batches in Echzeit zu analysieren.
        \item[Periodenanalyse] Es soll möglich sein für beliebige historische Zeiträume Geräte zu klassifizieren.
        \item[Performance] Es sollte möglich sein lange Zeitreihen mit sehr vielen Daten auch noch in annehmbarer Zeit auszuwerten. Zudem sollte es möglich sein Zeitreihen in Echtzeit zu analysieren.
        \item[Tests] Um Fehler zu verhindern und ein stabile und sicheres Programm zu entwickeln sollen wichtige fehleranfällige Funktionalitäten getestet werden.
    \end{description}


    Als Schnittstelle wird eine Socket.IO-API anstatt einer gebräuchlicheren REST-API aufgrund von besserer Performance und Einheitlichkeit mit der Daten-API gewählt.
    Die Spezifikation dieser Schnittstelle wird wie in Abschnitt \ref{API-Spezifikation} beschrieben, festgelegt. 
    Somit ist die Echzeitanalyse sowie die Periodenanalyse mit möglichst wenig Mehraufwand abgebildet.

    \subsection{API-Spezifikation}
        \subsubsection{Single Batch}
        \paragraph{Input:}

            \begin{lstlisting}[language=json,firstnumber=1]
Topic: "single-batch"
{
    "data": [
        {
            "u": 230,
            "f": 230,
            "h3": 230,
            "h5": 230,
            "h7": 230,
            "h9": 230,
            "h11": 230,
            "h13": 230,
            "h15": 230,
        },
        .. x Batchsize
    ]
}
            \end{lstlisting}
        
            \paragraph{Output:}
        
            \begin{lstlisting}[language=json,firstnumber=1]
Topic: 'single-prediction'
{
    "data": float //- Prediction of Senseo
}
            \end{lstlisting}
    
        \subsubsection{Period}
            \paragraph{Input:}
    
                \begin{lstlisting}[language=json,firstnumber=1]
Topic: "period"
{
    "data": {
        "start": "DateTime2",
        "end": "DateTime2"
    }
}
                \end{lstlisting}
            
                \paragraph{Output:}
            
                \begin{lstlisting}[language=json,firstnumber=1]
Topic: 'period-prediction'
* Series of Predictions where 0: Senseo, 1: Microwave, 2: Bosch, 3: Undefined, for every second
{
    "data": [
        Int,
        ... end - start
        Int
    ]
}
                \end{lstlisting}
