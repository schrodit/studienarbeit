\chapter{Grundlagen}

\section{Maschinelles Lernen (Machine Learning)}

    Wenn man Maschinelles Lernen oder Künstliche Intelligenz hört, denkt die Mehrzahl an Roboter mit eigenem Bewusstsein und Denken wie in vielen Science Fiction Filmen dargestellt.
    Jedoch ist Maschinelles Lernen mittlerweile keine Zukunftstechnologie mehr.
    Bereits in den 60er Jahren gab es erste Versuche der Wissenschaft Künstliche Intelligenz zu erschaffen.
    Doch was ist Maschinelles Lernen wirklich? Und was bedeutet es für einen Computer zu lernen?
    \newline

    \noindent
    Dieses Kapitel beschäftigt sich mit diesen Fragen und gibt einen kurzen Überblick über heutige Verfahren von Maschinellem Lernen.
    
    \subsection{Künstliche Intelligenz}
    Bevor Maschinelles Lernen erklärt werden kann sollte Künstliche Intelligenz im allgemeinen geklärt werden.


    \subsection{Einführung}
    Nimmt man den Begriff Maschinelles Lernen wörtlich beschreibt er das Lernen einer Maschine, also die Fähigkeit einer Maschine inteligenter zu werden.
    Von Maschinellem Lernen spricht man, falls eine Maschine auf Basis von Erfahrung und Fakten "ohne speziell programiert worden zu sein"\cite[20]{HandsOnML}, neues Wissen oder neue Zusammenhänge generieren kann.
    Wenn eine Maschine nachdem sie etwas gelernt hat bei der Ausführung einer Aktivität besser geworden ist, hat die Maschine maschinel gelernt\cite[20]{HandsOnML}.
    Das reine auswendig lernen von Fakten, wie beispielsweise das abspeichern einer Wikipedia-Seite auf die lokale Festplatte eines Computers, ist kein Wissenserwerb. 
    \newline

    \noindent
    Ein Beispiel für Maschinelles Lernen ist zum Beispiel der Spamfilter bei Emails. 
    Hier lernt ein Computer auf Basis von bisherigen Spammails neue Emails als Spam zu erkennen.
    
    \subsection{Lernprozess}
    Grundlegend besteht Maschinelles Lernen aus einer Trainigsphase und einer Test- beziehungsweise Validierungsphase.
    In der Trainigsphase wird auf Basis von Daten, wie bisherige Emails eines Benutzers, ein Modell erstellt, welches dann in der Validierungsphase auf seine Genauigkeit überprüft wird.
    Aus dem Ergebnis der Validierungsphase sowie der sonstigen Wissensbasis kann nun eine neue Trainsphase durchgeführt werden.
    \newline

    \noindent
    Dieser Prozess kann nun beliebig oft wiederholt werden und das Modell weiter verbessert werden.
    Das entstandene Modell ist das neu generierte Wissen.

    % Schaubild zu testphasen hinzufügen


    \subsection{Neuronale Nezte}
    


\section{Physikalische Grundlagen zur Netzaktivität} \label{physikalischeGrundlagen}

\section{Erhebung der Messdaten} \label{Messdaten}

    Um aussagekräftige Analysen und Klassifikationen über ein Stromnetz bzw. die Geräte in einem Stromnetz mit Maschinellem Lernen machen zu können, werden viele Trainings- und Testdaten benötigt.
    Die Daten bestehen aus verschiedenen physikalische Größen, die zu einem bestimmten Zeitpunkt in einem Stromnetz auftreten.
    Zu diesen Größen gehört die allgemeine Netzspannung, die Netzfrequenz sowie sieben harmonischen Oberwellen (vgl. \ref{physikalischeGrundlagen}).
    Um einen allgemeinen Überblick über den Verlauf der Netzaktivität zu erhalten sowie verschiedene Zeiten und Geräte vergleichen zu können, müssen Daten über lange Zeiträume erhoben werden.\\
    \newline
    Zur Erhebung der Werte zur Netzaktivität wurde ein WeSense-Messgerät\footnote{http://www.wesense-app.com/home-en/} verwendet.
    Dieses Gerät misst alle benötigten Werte und sendet diese über einen MQTT-Broker\footnote{Message Queuing Telemetry Transport} an einen Service von WeSense, welcher dann die Daten aufbereitet und in einer MSSQL Datenbank abspeichert.
    Die Werte werden sekündlich gemessen und in die Datenbank gespeichert, weshalb zunächst in eine row-based Datenbank gespeichert wird und später dann die Daten in eine column-based Datenbank zur schnellen Abfrage überführt werden.
    \newline

    \begin{figure}[h]
        \centering
        \includegraphics[width=1.0\textwidth]{Architecture}
        \caption{Complete Architecture}
        \label{fig:Architecture}
    \end{figure}

    \subsection{Klassifikation der Messdaten}
        
        Durch die oben beschriebene Erhebung sind die physikalischen Werte zu bestimmten Zeitpunkten bestimmt worden.
        Zusätzlich wird nun zur Identifikation der Geräte sowie zum Maschinellen Lernen, genau definierte Zeiträume benötigt in denen bestimmte Geräte aktiv waren.
        Dies bedeutet, dass jedem Zeitpunkt ein oder mehrere Geräte zugewiesen werden. \\
        \newline
        Um diese gelabelten Daten zu erheben gibt es verschiedene Möglichkeiten.
        Die Daten können entweder durch eine Person, welche Zeiten zu denen sicher Geräte aktiv waren manuell erfasst, oder durch eine Maschine automatisch erhoben werden.
        Jedoch wird zur automatischen Erhebung ein weiteres Gerät benötigt, welches zwischen dem zu messenden Gerät und dem Stromnetz zwischengeschalten wird und sobald Strom fließt Daten erfasst.
        Somit werden die manuell durch eine Person erfasst.\\
        \newline
        Hierzu wurde eine progressiv Web-App(vgl. Abbildung \ref{fig:WebApp1}) mit einer einfachen MySQL-Datenbank erstellt, mit der die Daten sehr einfach erfasst und abgespeichert werden können.

        \begin{figure}[h]
            \centering
            \includegraphics[width=0.5\textwidth]{WesenseConveyWebApp}
            \caption{Screenshot der progressive Web-App}
            \label{fig:WebApp1}
        \end{figure}
    
\section{Visualisierung}\label{VisualisierungWebApp}

        Zusätzlich zur manuellen Erhebung der Daten wurden zur besseren Analyse der Daten verschiedene Visualisierungsmöglichkeiten implementiert.
        Zum einen können die verschiedenen Physikalischen Größen eines gelabelten Gerätes zu einem bestimmten Zeitpunkt miteinander verglichen werden.
        Außerdem können bestimmte Größen zu verschiedenen gelabelten Zeiträumen eines Gerätes verglichen und analysiert werden. 
        Durch diese Visualisierung können sehr gut und genau Gemeinsamkeiten in verschiedenen Größen oder Zeiten erkannt werden.\\
        \newline
        Es werden verschiedene Diagramme sowie Normalisierungen der Daten zur Analyse bereitgestellt.
        Es besteht die Möglichkeit die Daten in einem Liniendiagramm sekündlich oder in frei wählbaren zusammengefassten Datenpunkten, sogennannten Klassen, anzuzeigen.
        Des weiteren können Histogramme mit verschiedenen Klassen gewählt werden.

        \begin{figure}[h]
            \centering
            \includegraphics[width=0.75\textwidth]{WesenseConveylineChart}
            \caption{Screenshot eines gelabelten Zeitraumes aus der Web-App}
            \label{fig:WebApp2}
        \end{figure}
        