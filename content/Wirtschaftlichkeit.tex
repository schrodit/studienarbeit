\chapter{Wirtschaftlichkeit}
    Das Ergebnis und die Erkenntnis aus dieser Arbeit kann in verschiedenen Industriebranchen, sowie auch im privaten Gebrauch eingesetzt werden.

    \paragraph{Hersteller von Haushaltsgeräten}
    Falls Hersteller von diversen Haushaltsgeräten auf anonymisierte Strommessdaten aus privaten Haushalten zugreifen können, wäre es möglich Laufzeiten ihrer Geräten zu analysieren.
    Somit kann die Aktivität, beziehungsweise die Benutzung dieser Geräte in Gebieten bestimmt werden. 
    
    Durch präzisere Verbraucherinformationen kann der \ac{CRM}-Prozess eines Unternehmens optimiert werden.
    Durch präzisere Messdaten können Fehlfunktionen klassifiziert werden und so "`Predictive Maintenance"' für die Geräte angeboten werden.

    \paragraph{Privater Gebrauch}
    Für den privaten Gebrauch kann die Aktivität der Geräte dadurch von Vorteil sein, dass intelligent Strom gespart werden kann.
    
    Auch kann eine Art digitaler Marktplatz eingerichtet werden, indem private Kunden Geräte klassifizieren und diese Daten an Unternehmen verkaufen können.

    \paragraph{Energieindustrie}
    In der Energieindustrie können mithilfe dieser Technik einfache Haushaltsgeräte erkannt werden. 
    Mit derselben Technik können auch Störungen sowie spezielle Merkmale eines Stromnetzes erkannt werden. 
    Dadurch kann analysiert werden wann und warum bestimmte Störungen auftreten, wodurch Energiezulieferer besser und schneller auf diese reagieren können.
    \newline

    Wie gezeigt wurde, kann das Ergebnis dieser Arbeit vielfältig in verschiedenen Wirtschaftzweigen eingesetzt werden.
    Um auf dem Markt erfolgreich zu sein, muss jedoch das Ergebnis dieser Arbeit mit mehr Daten trainiert und verbessert werden.
    Da dies jedoch für die Industrieunternehmen kein Problem darstellen sollte, kann diese Arbeit ein Denkanstoß sein.