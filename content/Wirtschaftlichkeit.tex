\chapter{Wirtschaftlichkeit}
    Das Ergebnis und die Erkenntnis aus dieser Arbeit kann in verschiedenen Industriebranchen sowie auch im privaten Gebrauch eingesetzt werden.

    \paragraph{Hersteller von Haushaltsgeräten}
    Falls Hersteller von diversen Haushaltsgeräte, wie einer Kaffeemaschine, auf anonymisierte Strommessdaten von Messgeräten, welche in z.B. in Stromverteilern von Häusern fest verbaut sind, können verschiedene Analysen auf diesen Daten gemacht werden.
    Es könnte zum Beispiel interessant sein, wie viele Geräte in einem Bezirk wann aktiv sind. 
    So könnte der \ac{CRM}-Prozess optimiert werden.
    Auch kann bei genügend Messdaten, welche von Herstellern sehr einfach erhoben werden können, Fehlfunktionen festgestellt werden und so "`Predictive Maintenance"' für deren Geräte betrieben werden.

    \paragraph{Privater Gebrauch}
    Für den privaten Gebrauch könnte es interessant sein zu welchen Zeiten bestimmte Geräte aktiv waren und Stromverbraucht haben.
    Dadurch kann festgestellt werden ob unerwünschte Geräte aktiv waren oder anderweitig Strom gespart werden kann.
    
    Auch kann eine Art digitaler Marktplatz eingerichtet werden, indem private Kunden Geräte klassifizieren und diese Daten an Unternehmen verkaufen können.

    \paragraph{Energieindustrie}
    In der Energieindustrie können mithilfe dieser Technik nicht nur einfache Haushaltsgeräte erkannt werden. 
    Mit derselben Technik können auch Störungen sowie spezielle Merkmale erkannt werden. 
    Dadurch kann analysiert werden wann und warum bestimmte Störungen auftreten und kann auf diese Besser reagieren oder beseitigen.
    \newline

    Wie gezeigt wurde, kann das Ergebnis dieser Arbeit vielfältig in verschiedenen Wirtschaftzweigen eingesetzt werden.
    Um auf dem Markt erfolgreich zu sein, muss jedoch das vorhandene Modell mit mehr Daten trainiert und verbessert werden.
    Da dies jedoch für größere Unternehmen kein Problem sein sollte, kann diese Arbeit ein Denkanstoß für diese Unternehmen sein.